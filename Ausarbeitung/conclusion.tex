%!TEX root = main.tex
\newpage
\section{Conclusion}
\label{sec:conclusion}
The verification method for multiply linked data structures described above has been implemented in a Java prototype.
This implementation though has been extended with a light-weight flow-based alias analysis to prune state space.
Results from several experiments with this prototype can be seen in table \ref{Results}. The table shows the time it took to run the analysis as well
as the number of signatures computed throughout the analysis. These results show that the proposed method does indeed work but
it has to be taken with a small amount of concern since the sample size is really small. Further optimizations, improvements and tests
have to be done to give these results more meaning.\\
Such improvements could for example be parallelization to reduce
the time used for analysis or checking the entailment on the heap
signatures.
On the other hand has this approach a fundamental strength which should be mentioned. The method is very generic and can therefore be 
used for analysis on other more complex data structures as well, if extended accordingly. This provides a good amount of future work that can be done like for example using this method on skip lists.

\begin{figure}[htb]
\caption{Experimental Results}
\label{Results}
\begin{center}
\begin{tabular}{|c|c|c|c|}

\hline
Program & Struct & Time & \#Sig \\
\hline 
Traverse & DLL & 11.4 s & 294 \\
\hline
Insert & DLL & 3.5 s & 121 \\
\hline
Ordered Insert & DLL & 19.4 s & 793 \\
\hline
Merge & DLL & 6 min 40 s & 8171 \\
\hline
Reverse & DLL & 10.8 s & 395 \\
\hline
Search & Tree & 1.2 s & 51 \\
\hline
Insert & Tree & 6.8 s & 241 \\
\hline
\end{tabular}
\end{center}
\end{figure}




\begin{comment}
Give a conclusion on your topic. Give a few sentences to summarize 
the topic. If possible, point out the quality of the result and give
a small prospect of subsequent works.
\end{comment}