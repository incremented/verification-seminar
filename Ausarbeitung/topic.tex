%!TEX root = main.tex
\newpage
% Heap
\section{Heap}

% Heap Introduction
To model the memory used by the program we want to verify, we use a graph with nodes and labeled edges.
Each node represents a memory cell whereas each edge represents a pointer to the next part of the data-structure.
The edges are labeled with a color to distinguish between these selectors. For simplicity only structures with
two pointers will be considered. The selectors will be labeled 1 and 2 
(instead of common known selectors \textit{next} and \textit{prev}). The approach we use though can be generalized
for structures with any number of selectors.\\
We now need to define two special nodes, one for the \textit{null pointer} and one for \textit{dangling pointers}.
The \textit{null node}, written as $\#$, will be used to model null successors whereas the \textit{dangling node},
written as $*$, will be used to model pointers which used to point to memory that has been either freed or not yet been allocated.
Both these nodes will be used by making the corresponding edge link to one of them.\\
In addition, we need to model a program variable by labeling the node the variable points to, with the variable's name.\\
The above leads to the following formal definition of a \textit{heap}.\\
Let $C = \{1,2\}$ be a set of colors and $X$ be a set of program variables.\\

% Heap Definition
$heap = (\overline{M}, E, s, t, \tau, \lambda)$ where:

\begin{itemize}
	\item $\overline{M} = M \cup \{\#,*\}$ represents the finite set of allocated memory cells, together with the two special nodes 
		representing the null value and the dangling pointer, respectively.
	\item $E$ is a finite set of edges.
	\item The source function $s : E \rightarrow M$ is a total function that gives the source of the edges.
	\item The target function $t : E \rightarrow M$ is a total function that gives the target of the edges.
	\item The type function $\tau : E \rightarrow C$ is a total function that gives the color of the edges.
	\item $\lambda : X \rightarrow M$ is a total function that defines the positions of the program variables.
\end{itemize}

% TODO: References
\noindent
% Heap Invariant
To verify that each memory cell has exactly one outgoing edge labeled 1 and one outgoing edge labeled 2, the heap must satisfy
the following invariant:\\

$\forall c \in C$ $\forall m \in M : |s^{-1}(m) \cap \tau^{-1}(c)| = 1$ \\

\noindent
% Heap operations
\subsection{Operations on Heaps}
To work with these heaps, we have to define two operations on it. \\
Let $h = (\overline{M}, E, s, t, \tau, \lambda)$ be a heap, 
then $h \oplus m$ is equal to $h'$ for $m \in M$, where $h'$ is the heap resulting from the addition of the cell $m$ and it's two
outgoing \textit{dangling} edges.\\
Respectively $h \ominus m$ is defined as the heap $h'$ with the cell $m$ and it's outgoing edges removed, where
all edges that previously pointed to $m$ are now \textit{dangling} edges.
% TODO: Formal Definition if enough space left

\newpage
% Signatures
\section{Signatures}
We now define a way to characterize heaps into classes with certain properties.
Therefore we introduce signatures, which can be seen as a predicate describing a set of minimal
conditions a heap has to fulfill to satisfy the predicate. 
Intuitively a signature can be viewed as a heap with some parts "missing" \cite{abdulla2013monotonic}. 

\noindent
We define signatures as a tuple $(\overline{M}, E, s, t, \tau, \lambda)$ identically to heaps with the exception,
that the functions $\tau$ and $\lambda$ can be partial. As for the heaps, we also define two invariants for
signatures:

% Signature Invariants
\begin{enumerate}
	\item $\forall c \in C$ $\forall m \in M : |s^{-1}(m) \cap \tau^{-1}(c)| \le 1$
	\item $\forall m \in M : |s^{-1}(m)| \le 2$
\end{enumerate}

\noindent
The first invariant states, that each signature's memory cell can't have more than one outgoing edge of each color.
Whereas the second invariant states, that each memory cell can have at most two outgoing edges in total.

\subsection{Operations on Signatures}
As we did for the heaps, we define operations for signatures. Since signatures are basically heaps, we
can carry over the definition of $\ominus$.\\
Let now $sig = (\overline{M}, E, s, t, \tau, \lambda)$ be a signature. 
We define $sig \boxminus e$ as the signature $sig'$ where $e \in E$ has been removed from $sig$ and it's functions 
$s$, $t$ and $\tau$ are restricted to $E' = E\setminus\{e\}$. We define the addition of an edge in a similar way.
Given $m_1 \in M$, $m_2 \in \overline{M}$ and $c \in C$, $sig \boxplus (m_1 \xrightarrow{c} m_2)$ denotes the addition
of a transition of color $c$ between $m_1$ and $m_2$ to $sig$. We conclude with the definition for the addition of
a memory cell $m \not\in \overline{M}$ as follows: $sig.(\overline{M} := \overline{M} \cup {m})$ equals the signature
$sig' = (\overline{M} \cup {m}, E, s, t, \tau, \lambda)$.

\subsection{Ordering on Signatures}

To define an ordering on signatures, we need to take a look at the following five $ordering$ $steps$ and how they are defined first.
We begin with the definition of $unlabeled$, $isolated$ and $simple$ memory cells.
Given a signature $sig = (\overline{M}, E, s, t, \tau, \lambda)$ and a cell $m \in \overline{M}$, $m$ is $unlabeled$ if 
$\lambda^{-1}(m) = \emptyset$. We say the cell is $isolated$, if it is $unlabeled$ and also $s^{-1}(m) = \emptyset$ and 
$t^{-1}(m) = \emptyset$ hold. The cell $m$ is called $simple$, when $m$ is $unlabeled$ and the following constraints
hold: $s^{-1}(m) = \{e_1$\}, $t^{-1}(m) = \{e_2\}$, $e_1 \not= e_2$ and $\tau(e_1) = \tau(e_2)$. 
Given signatures $sig_1 = (\overline{M}_1, E_1, s_1, t_1, \tau_1, \lambda_1)$ and
$sig_2 = (\overline{M}_2, E_2, s_2, t_2, \tau_2, \lambda_2)$, we can now define the application of an $ordering$ $step$
written as $sig_1 \vartriangleleft sig_2$ if $sig_1$ results
from the application of one of the following operations:

\begin{itemize}

	\item $Isolated$ $cell$ $deletion$. There is an isolated $m \in M_2$ s.t. $sig_1 = sig_2 \ominus m$.
	\item $Edge$ $deletion$. There is an edge $e \in E_2$ such that $sig_1 = sig_2 \boxplus e$.
	\item $Contraction$. There is a simple cell $m \in M_2$, edges $e_1,e_2 \in E_2$ with $t_2(e_1) = m$,
		  $s_2(e_2) = m$, $\tau(e_1) = \tau(e_2)$, and $sig_1 = sig_2.t[e_1 \mapsto t(e_2)] \ominus m$. 
	\item $Edge$ $decoloring$. There is an edge $e \in E_2$ such that $sig_1 = sig_2.\tau[e \mapsto \bot]$
	\item $Label$ $deletion$. There is a label $x \in X$ such that $sig_1 = sig_2.\lambda[x \mapsto \bot]$

\end{itemize}
% TODO: Reference!
\noindent
This leads us to the definition of $ordering$ on signatures. We write $sig_1 \vartriangleleft \vartriangleleft sig_2$ to denote
that a signature $sig'$ exists such that $sig_1 \vartriangleleft sig'$ and $sig' \vartriangleleft sig_2$.
A signature $sig_1$ is now called $smaller$ than a signature $sig_2$, written as $sig_1 \sqsubseteq sig_2$, if there is
a sequence of $ordering$ $steps$ from $sig_2$ to $sig_1$. The $\sqsubseteq$-relation used here, is formally the reflexive transitive
closure of $\vartriangleleft$.\\
With the $ordering$ on signatures defined, we can now take a look at the semantics of signatures.
To say that a Heap $h$ satisfies a signature $sig$ we write $h \in \llbracket sig \rrbracket$, if $sig \sqsubseteq h$.
Where $\llbracket sig \rrbracket$ represents the set of all heaps in the $upward$ $closure$ of $sig$ with respect to our
$ordering$-relation. We define a set of signatures $S$ in the same way as 
$\llbracket S \rrbracket = \bigcup_{s \in S} \llbracket s \rrbracket$.

\subsection{Bad Configurations}
We will now use our signatures defined above to specify configurations of $heaps$ which are not correct. 
Our intention is to specify a finite set of signatures describing all configurations which we consider as $bad$ $states$.
To keep this paper as short as possible, we only take a look at one example of a $bad$ $state$ and only refer to the complete
list of $bad$ $states$ \cite{abdulla2013monotonic}.\\
% TODO: Reference!
Example of a $bad$ $state$:

\begin{itemize}
	
	\item Non-Cyclicity: This $bad$ $state$ refers to all structures which have a selector pointing to the same memory cell $m$.
		We can easily give a set of $signatures$ for this with $S = \{s_1, s_2\}$ where $s_1$ and $s_2$ are defined as below.

	\begin{center}
		\begin{tikzpicture}
		  [scale=.8,auto=left,main node/.style={circle,fill=blue!20}, background rectangle/.style={fill=gray!20}, show background rectangle]
		  \node[main node] (1) {};
		  \node[text width=1cm, anchor=ost, left] at (0,0) {$s_1:$};
		  \path (1) edge [loop above] node {1} (1);
		\end{tikzpicture}
		\qquad
		\begin{tikzpicture}
		  [scale=.8,auto=right,main node/.style={circle,fill=blue!20}, background rectangle/.style={fill=gray!20}, show background rectangle]
		  \node[main node] (1) {};
		  \node[text width=1cm, anchor=ost, left] at (0,0) {$s_2:$};

		  \path (1) edge [loop above] node {2} (1);
			\end{tikzpicture}
	\end{center}
\end{itemize}
% TODO: edges thicker

\subsection{Transition System}
We conclude the definition part with a short introduction to Transition-Systems. A Transition-System $(S, \longrightarrow)$ consists
of a set of states $S$ and the transition relation $\longrightarrow$. A state of such a transition system is a pair $(pc, h)$ where
$h$ is a heap and $pc$ is the current Program-Counter in the program running. Given states $s = (pc, h)$ and $s' = (pc', h')$, we 
say a transition from $s$ to $s'$ exists, written as $s \longrightarrow s'$, if there is a transition, such that 
$h \xrightarrow{op} h'$. In other words, there is a heap operation defined, which changes the heap $h$ into $h'$. In our case,
we use a subset of the C-Programing Language and we only consider the following operations:
\begin{itemize}
	\item $op$ is of the form $x == y$
	\item $op$ is of the form $x$ $!= y$
	\item $op$ is of the form $x = y$
	\item $op$ is of the form $x = y.next(i)$
	\item $op$ is of the form $x.next(i) = y$
	\item $op$ is of the form $x = malloc()$
	\item $op$ is of the form $free(x)$
\end{itemize}

\section{Monotonic Abstraction}
This section will introduce the concept of $monotonic$ $abstraction$. We call a transition system $(S, \longrightarrow)$ with
the ordering relation $\sqsubseteq$ we defined earlier $monotonic$, if the following holds. For any given $states$ $sig_1$, $sig_2$
and $sig_3$ such that $sig_1 \sqsubseteq sig_2$ and $sig_1 \longrightarrow sig_3$, we can always find a state $sig_4$ such that
$sig_2 \longrightarrow sig_4$ and $sig_3 \sqsubseteq sig_4$.
% TODO: Reference!
The problem is that the transition system we defined before is not $monotonic$. We therefore have to construct an over-approximation
for this system in such a way that if the over-approximation does not contain any reachable $bad$ $states$, we can conclude that
the original system does also not contain any reachable $bad$ $states$. We do this by redefining the transition-relation as follows.
Let $\longrightarrow$ be the original transition-relation, then $\longrightarrow_A$ is the transition-relation $\longrightarrow$
in such a way that it becomes $monotonic$. We can construct $\longrightarrow_A$ by using state $sig_3$ from the monotonic defintion
as the $sig_4$ required by the definition. Or more formally, $s \longrightarrow s'$ iff there is an $s''$ such that $s'' \sqsubseteq s$
and $s'' \longrightarrow s'$. There is a problem with this approach though since it only holds in one direction. 
This may generate false alarms during the analysis which however did not occur in the experiments given. 

\subsection{Computing Predecessors}
The following section is about the relation $pre$ used to determine all predecessors of a given signature $sig$ with respect to
our abstraction of the transition-relation $\longrightarrow_A$. In order to compute this relation, we have to define auxiliary 
operations on signatures. These auxiliary operations, namely $add$ $variable$, $add$ $edge$ and $add$ $label$ are procedures
which add the given $component$ $(variable, edge, label)$ to all places where the $component$ might occur in a heap $h$ such that
$h$ still satisfies this signature. \\
We begin with the $add$ $variable$ operation.\\
Given $M^{\#} = M \cup \{\#\}$ and $sig = (\overline{M}, E, s, t, \tau, \lambda)$. We define the set 
$sig \uparrow(\lambda(x) \in M^{\#})$ to be the set of all signatures $sig'=(\overline{M'}, E', s', t', \tau', \lambda')$ 
so that one of the following is true:

\begin{itemize}
	\item $\lambda(x) \in M^{\#}$ and $sig = sig'$. In this case, the variable is already present and we don't have to do somethin.
	\item $\lambda(x) = \bot$ and $sig \vartriangleleft_x sig'$. We add $x$ to a cell that is explicitly represented in $sig$.
	\item $\lambda(x) = \bot$ and $sig \vartriangleleft_{\lambda'(x)} \vartriangleleft_x sig'$. We add $x$ to a cell that is missing in $sig$. Note that according to the definition of 
	$\llbracket sig \rrbracket$, there may exist cells in $h \in \llbracket sig \rrbracket$ that are not explicitly represented in $sig$.
\end{itemize}

\noindent
The second operation we want to define is $add$ $edge$, which adds an edge between two cells in a signature $sig$. Let $m_1 \in M$ and $m_2 \in \overline{M}$, a signature $sig'$ is now in the set
$sig\uparrow(m_1 \xrightarrow{e} m_2)$ if one of the following holds:

\begin{itemize}
	\item There is an edge $e \in E$ such that $s(e) = m_1$, $t(e) = m_2$, $\tau(e) = c$ and $sig' = sig$. 
		This is the simple case where we do not have to do anything.
	\item There is an edge $e \in E$ such that $s(e) = m_1$, $t(e) = m_2$, $\tau(e) = \bot$, there is no $e' \in E$ such that
		$s(e') = m_1$ and $\tau(e') = c$, and we have a $sig' = sig.\tau [e \mapsto c]$. We have a decolored edge which we can update.
	\item There is no $e \in E$ such that $s(e) = m_1$ and $\tau(e) = c$, $|s^{-1}(m_1)| \le 1$ 
	and $sig' = sig \boxplus (m_1 \xrightarrow{c} m_2)$. The edge is not present and $m_1$ has no outgoing edge of this color.
\end{itemize}
% TODO: Reference!

\noindent
The third and last operation that we need to define is the $add$ $label$ operation which will add two labels $x$ and $y$ to the
given signature $sig$ so that they both label the same cell.
We say that a signature $sig'$ is in the set of signatures 
$sig\uparrow(\lambda(x) = \lambda(y))$, if one of the following
holds:

\begin{itemize}
	\item $\lambda(x) \in M^{\#}$, $\lambda(x) = \lambda(y)$ and $sig' = sig$. Again, this is the simple case where both labels are already present.
	\item $\lambda(x) = \bot$, $\lambda(y) \in M^{\#}$ and $sig' = sig.\lambda[x \mapsto \lambda(y)]$. The given label $x$ is not yet present so we add it to the cell labeled by $y$.
	\item $\lambda(y) = \bot$ and there is a $sig_1 \in sig\uparrow(\lambda(x) \in M^{\#})$ such that $sig' = sig_1.\lambda[y \mapsto \lambda_1(x)]$. Since the label $y$ is not yet present, we add it to a signature where $x$ is guaranteed to be present.
\end{itemize}

\noindent
With these auxiliary operations defined, we can now begin with defining the actual predecessors function. Given a signature $sig$
and an operation $op$, the set of all predecessors is written as $pre(op)(sig)$. In the following, we will define $pre$ for each 
operation we defined earlier, by giving a set of constraints, each signature that is in the set of predecessors has to satisfy.
We begin with the simpler operations first.\\
Let $sig = (\overline{M}, E, s, t, \tau, \lambda)$ be a signature. We say that the set of signatures \\
$sig' = pre(x = y)(sig)$ is the set of signatures such that there exists a signatures $sig_1$ satisfying the following constraint.

\begin{itemize}
	\item $sig_1 \in sig\uparrow(\lambda(x) = \lambda(y))$ and $sig_1 \vartriangleleft_x sig'$.
\end{itemize}

\noindent
We define $pre(x == y)(sig)$ is in the same way as $sig \uparrow(\lambda(x) = \lambda(y))$.\\
To define $pre(x$ $!= y)(sig)$ we take a similar approach. We say that $pre(x$ $!= y)(sig)$ is the set of all signatures
$sig' = (\overline{M'}, E', s', t', \tau', \lambda')$ with $\lambda'(x) \not=\lambda'(y)$ satisfying:

\begin{itemize}
	\item There is no signature $sig_1 \in sig\uparrow(\lambda(x) \in M^{\#})$ such that $sig' \in sig_1\uparrow(\lambda(y)\in M^{\#}$.
\end{itemize}

\noindent
For $pre(x = y.next(i))(sig)$ we need a slightly different approach.\\
We say that $pre(x = y.next(i))(sig)$ is defined as the set
of signatures $sig' = (\overline{M'}, E', s', t', \tau', \lambda')$ such there exist signatures $sig_1$, $sig_2$, and $sig_3$ which
satisfy the following constraints.

\begin{itemize}
	\item $sig_1 = sig\uparrow(\lambda(x) \in M^{\#})$
	\item $sig_2 = sig_1\uparrow(\lambda(y) \in M^{\#}$
	\item $sig_3 \in sig\uparrow(\lambda_2(y) \xrightarrow{i} \lambda_2(x))$
	\item $sig' = sig_3.\lambda[x \mapsto \bot]$
\end{itemize}

\noindent
In the same manner as for  $pre(x = y.next(i))(sig)$ we will now define\\
$pre(x.next(i) = y)(sig)$ as the set of all signatures
$sig' = (\overline{M'}, E', s', t', \tau', \lambda')$ where signatures $sig_1$, $sig_2$, $sig_3$ and an edge $e \in E_3$ exists
such that these constraints hold:

\newpage

\begin{itemize}
	\item $sig_1 = sig\uparrow(\lambda(x) \in M^{\#})$
	\item $sig_2 = sig_1\uparrow(\lambda(x) \in M^{\#})$
	\item $sig_3 \in sig\uparrow(\lambda_2(x) \xrightarrow{i} \lambda_2(y))$
	\item $s_3(e) = \lambda_3(x)$, $t_3(e) = \lambda_3(y)$, $\tau(e) = i$
	\item $sig' = sig_3 \boxminus e$
\end{itemize}

\noindent
To conclude the predecessors calculation, we define the functions for memory allocation and deallocation.
Let $sig$ be a signature. We say $pre(x = malloc())(sig)$ is the set $sig'$ of signatures, such that there exist signatures
$sig_1$, $sig_2$ and $sig_3$ satisfying:

\begin{itemize}
	\item $sig_1 \in sig\uparrow(\lambda(x) \in M^{\#})$, $\lambda_1(x) \not= \#$, and there exists no $y \in X \setminus \{x\}$
	such that $\lambda_1(y) = \lambda_1(x)$, $t^{-1}(\lambda(x)) = \emptyset$, and for every edge $e \in E_1$ such that
	$s_1(e) = \lambda_1(x)$ the following holds: $t_1(e) = *$.
	\item $sig_2 = sig_1 \ominus \lambda_1(x)$
	\item $sig' = sig_2.\lambda[x \mapsto \bot]$
\end{itemize}

\noindent
We end this section with the definition for memory deallocation.
Given a signature $sig$, we say that $pre(free(x))(sig)$ is the set $sig'$ of all signatures where signatures $sig_1$ and $sig_2$
with $m \not\in \overline{M_1}$ exists, which satisfy these constraints:

\begin{itemize}
	\item $\lambda(x) = *$
	\item $\overline{M_1} = \overline{M}$, $E_1 = E \setminus t^{-1}(*)$, $s_1 = s|_{E_1}$, $t_1 = t|_{E_1}$, $\tau_1 = \tau|_{E_1}$
	and for every $y \in X$, $\lambda_1(y) = \bot$ $if$ $\lambda(y) = *$, $\lambda_1(y) = \lambda(y)$ otherwise
	\item $sig_2 = sig_1.(\overline{M} := \overline{M} \cup \{m\})$
	\item $sig' = sig_2.\lambda[x \mapsto m]$
\end{itemize}

\noindent
With the predecessor function now completely defined, we can take a look at the Reachability Algorithm which uses $pre$. 
I would like to mention though, that we did not have proven the correctness of the $pre$ function yet. The prove for the correctness
with respect to the abstract transition relation $\longrightarrow_A$ can be found in the main reference of this seminar-paper \cite{abdulla2013monotonic}.
% ausformulieren vielleicht

\section{The Reachability Algorithm}
In this section, we will describe how the Backward-Reachability-Algorithm works. The algorithm is given a set of $bad$ $states$ 
$S_{bad}$ as defined earlier and computes all successive sets $s_0, S_1, S_2 ...$, where $S_0 = S_{bad}$ and
$S_{i + 1} = \bigcup_{s \in S_i} pre(s)$. If during this process, a signature $s$ is being generated such that there exists a signature
$s'$ which has been generated before, with $s' \sqsubseteq s$, the signature $s$ can be omitted from further analysis. The analysis
finishes, when all new generated signatures are discarded. At this point, all generated signatures left denote exactly these heaps,
which can reach a $bad$ $state$ using our approximated transition relation $\longrightarrow_A$. If both these sets are disjoint, 
the safety property holds, otherwise it fails. 

\begin{comment}
This section concerns the main topic. In the following you can see a
small illustration of how to use itemizings and enumerations.


\begin{itemize}
	\item Point 1.
	\item Point 2.
\end{itemize}

\begin{enumerate}
	\item Point 1.
	\item Point 2.
	\begin{enumerate}[I)]
  \item Point 1.
  \item Point 2.
\end{enumerate}
\end{enumerate}

\begin{enumerate}
	\item Point 1.
	\item Point 2.
\end{enumerate}

\begin{description}
	\item[Term one: ] Description of term one.
	\item[Term two: ] Description of term two.
\end{description}

In Algorithm~\ref{alg:nameOfAlgorithm} you can see how we define an
algorithm.

\begin{Algorithm}
	\caption{Describe the purpose of the algorithm. For more 
	information see the \href{../manuals/newalg.pdf}{newalg-
	Manual}.}
	\label{alg:nameOfAlgorithm}
	\begin{algorithm}{\text{void method}}{\text{typeA argumentA, 
												typeB argumentB}}
		\text{write the algorithm in pseudocode} \\
		\text{it should not go into detail, but display main idea} \\
		\text{however, keep being consistent} \\
		x\=1\text{ (this is how to assign a value to a variable)} \\
		\begin{WHILE}{\text{a condition being True or False}}
			\text{do something} \\
			\text{and something else} \\
		\end{WHILE} \\
		\begin{IF}{\text{a condition being True or False}}
			\text{point }1 \\
		\ELSE
			\begin{IF}{\text{another condition}}
				\text{point }2 \\
			\end{IF}
		\ELSE
			\text{point }3 \\
		\end{IF}
		\RETURN True
	\end{algorithm}
\end{Algorithm}

\subsection{Example}
Give an example to illustrate the idea of your topic. Import images
in the following way. Store the images in a separate folder as 
precasted in our template.

\begin{figure}[htb]
	\begin{center}
		\includegraphics[width=0.8\linewidth]{pictures/ratte.jpg}
	\end{center}
	\caption{Proseminar supervisor's pet.}
	\label{fig:rat}
\end{figure}

\end{comment}