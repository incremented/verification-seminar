%!TEX root = ./main.tex
\begin{frame}[fragile]{Signatures}
    
    \begin{itemize}
        \item <2-> Defined in the same way as heaps except for:
        \item <3-> The \textit{type function} and the \textit{variable function} can be partial
        \item <4-> Less strict invariants: \\
            \begin{itemize}
                \item <5-> $\forall c \in C$ $\forall m \in M : |s^{-1}(m) \cap \tau^{-1}(c)| \le 1$
                \item <6-> $\forall m \in M : |s^{-1}(m)| \le 2$
            \end{itemize}
        \item <7-> Signatures are partial heaps
        \item <8-> What does a Signature actually represent?
    \end{itemize}

\end{frame}

\begin{frame}[fragile]{Signatures}
    
    \pause

    \begin{itemize}
        \item <2-> Signatures represent infinite sets of heaps
        \item <3-> Or in other words:
    \end{itemize}

    \pause 

    \pause

    \medskip

     A signature is a heap with some parts 'missing'. It represents the set of all heaps that have at least the property or structural information described by this signature.
\end{frame}

\begin{frame}[fragile]{Ordering on Signatures}
    
    \pause

    Ordering Steps:

    \pause

    \begin{itemize}
        \item <3-> Isolated cell deletion
        \item <4-> Edge deletion
        \item <5-> Contraction
        \item <6-> Edge decoloring
        \item <7-> Label deletion
    \end{itemize}

\end{frame}

\begin{frame}[fragile]{The Ordering Relation}
    
    \pause

    \begin{definition}
        We say a signature $sig_1$ is smaller than a signature $sig_2$, if there is a sequence of 
        ordering steps that leads from $sig_2$ to $sig_1$, written as $sig_1$ $\sqsubseteq$ $sig_2$.
    \end{definition}

    \pause

    \begin{itemize}
        \item <3-> We say that a heap $h$ satisfies $sig_1$, written as $h \in \llbracket sig_1 \rrbracket$, if $sig_1 \sqsubseteq h$
    \end{itemize}

\end{frame}

\begin{frame}[fragile]{Bad Configurations}
    
    \pause

    \begin{itemize}
        \item <2-> Heap Configurations that should \textbf{never} occur
        \item <3-> Represented as signatures
        \item <4-> Example of a Bad Configuration:
    \end{itemize} 

    \pause 

    \pause

    \pause

    \medskip
    \medskip

    \medskip

    Non-Cyclicity: This bad configuration refers to all structures which have a selector pointing to the same memory cell $m$.

    \pause

    \begin{center}
        \begin{tikzpicture}
            [scale=.8,auto=left,main node/.style={circle,fill=blue!20}, background rectangle/.style={fill=gray!20}, show background rectangle]
            \node[main node] (1) {};
            \node[text width=1cm, anchor=ost, left] at (0,0) {$s_1:$};
            \path (1) edge [loop above] node {1} (1);
            \end{tikzpicture}
            \qquad
            \begin{tikzpicture}
            [scale=.8,auto=right,main node/.style={circle,fill=blue!20}, background rectangle/.style={fill=gray!20}, show background rectangle]
            \node[main node] (1) {};
            \node[text width=1cm, anchor=ost, left] at (0,0) {$s_2:$};

            \path (1) edge [loop above] node {2} (1);
        \end{tikzpicture}
    \end{center}
\end{frame}
